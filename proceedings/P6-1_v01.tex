
\documentclass[11pt,twoside]{article}

% Do NOT use ANY packages other than asp2014. 
\usepackage{asp2014}

\aspSuppressVolSlug
\resetcounters

% References must all use BibTeX entries in a .bibfile.
% References must be cited in the text using \citet{} or \citep{}.
% Do not use \cite{}.
% See ManuscriptInstructions.pdf for more details
\bibliographystyle{asp2014}

% The ``markboth'' line sets up the running heads for the paper.
% 1 author: "Surname"
% 2 authors: "Surname1 and Surname2"
% 3 authors: "Surname1, Surname2, and Surname3"
% >3 authors: "Surname1 et al."
% Replace ``Short Title'' with the actual paper title, shortened if necessary.
% Use mixed case type for the shortened title
% Ensure shortened title does not cause an overfull hbox LaTeX error
% See ASPmanual2010.pdf 2.1.4  and ManuscriptInstructions.pdf for more details
\markboth{Ruiz et al.}{Versioned executable user documentation for in-development science tools}

\begin{document}

\title{Versioned executable user documentation for in-development science tools}

% Note the position of the comma between the author name and the 
% affiliation number.
% Author names should be separated by commas.
% The final author should be preceded by "and".
% Affiliations should not be repeated across multiple \affil commands. If several
% authors share an affiliation this should be in a single \affil which can then
% be referenced for several author names.
% See ManuscriptInstructions.pdf and ASPmanual2010.pdf 3.1.4 for more details
\author{J. E. Ruiz$^1$, C. Boisson$^2$, C. Deil$^3$, A. Donath$^3$ and B. Khelifi$^4$}
\affil{$^1$Instituto de Astrof\'isica de Andaluc\'ia - CSIC, Granada, Spain; \email{jer@iaa.es}}
\affil{$^2$LUTH - Observatoire de Paris, Paris, France}
\affil{$^3$Max-Planck-Institut f\"ur Kernphysik, Heidelberg, Germany}
\affil{$^4$APC - AstroParticule et Cosmologie, Universit\'e Paris Diderot, Paris, France}


% This section is for ADS Processing.  There must be one line per author.
\paperauthor{os\'e Enrique Ruiz}{jer@iaa.es}{0000-0003-3274-4445}{Instituto de Astrof\'isica de Andaluc\'a - CSIC}{Departmento de Astronom\'ia Extragal\'actica}{Granada}{Spain}{18008}{Spain}


\begin{abstract}
One key aspect of software development is feedback from users. This community is not always aware of the
modifications made in the code base, neither they use the tools and practices followed by the developers to deal with
a non-stable software in continuous evolution.  The open-source Python package for gamma-ray astronomy Gammapy, provides 
its user community with versioned computing environments and executable documentation, in the form of Jupyter
notebooks and virtual environment technologies that are versioned coupled with the code base. We find that this 
set-up greatly improves the user experience for a software in prototyping phase, as well as provides a good workflow to
maintain an up-to-date documentation.
\end{abstract}

% These lines show examples of subject index entries. At this stage these have to commented
% out, and need to be on separate lines. Eventually, they will be automatically uncommented
% and used to generate entries in the Subject Index at the end of the Proceedings volume.
% Don't leave these in! - replace them with ones relevant to your paper.
%\ssindex{conference!ADASS 2017}
%\ssindex{organisations!ASP}

\section{The Gammapy package}

Gammapy\footnote{\url{https://gammapy.org}} \citep{2017ascl.soft11014D} is a community-developed, open-source Python package for high-level $\gamma$-ray data analysis, built on Numpy \citep{oliphant_guide_2006} and Astropy \citep{2013ascl.soft04002G}. It provides functionalities to create sky images, spectra, light curves and source catalogs from event lists and instrument response information, determining the position, morphology and flux of $\gamma$-ray sources. Gammapy is a prototype for the Cherenkov Telescope Array (CTA) science tools. It has been used to simulate and analyse data for the CTA, as well as for the main IACT (Imaging Air Cherenkov Telescopes) facilities like H.E.S.S., MAGIC, VERITAS,  and the Fermi-LAT telescope.  

\section{The user role for \textit{in-development} software}

The Gammapy package is a software in evolution, rapidly changing, where many developments are ongoing. As such, it is a place for $\gamma$-ray astronomers to share their code and contribute. So far, developer coding sprints have been scheduled to happen with the same regularity of user hands-on sessions held in CTA consortium meetings. A small collection of tutorials, in the form of Jupyter notebooks \citep{Kluyver:2016aa}, was originally provided for learning purposes as a complement to the web published documentation. The documentation and notebooks had to be updated frequently and separately. With the rise in the contributor and user base, as well as in the development activity, it emerged the need to have the notebooks integrated in the documentation, and versioned coupled with the code base. Users should easily bring to life the different versions of the published tutorials, so they could be able reproduce and re-use them for their own goals.

\section{Integrating versioned and executable user documentation}

A graphic description of the technical set-up may be found in Figure \ref{fig1}. The source files for the documentation, the Jupyter notebooks, and the code base are maintained in a single GitHub repository\footnote{\url{https://github.com/gammapy/gammapy}}. The code base and notebooks are continuously built and tested in different environments using the continuous integration service Travis CI.

The notebooks are stored stripped of their output cells, which greatly helps in identifying differences in the contribution review. We have implemented our own solution for execution validation of the notebooks: these are executed with \textit{jupyter nbconvert}, the output cells parsed and, in case an error is found in one of the output cells, validation fails. Code formatting of their input cells is also done parsing their content with the help of the \textit{Black} package.

During the documentation building process, the stripped notebooks are executed, so to refill their output cells, and later merged into the documentation with Sphinx and the \textit{nbsphinx} extension. It is possible to skip the notebooks integration, since the execution step may be quite time consuming. In this sense, only fast simple notebooks are chosen to be part of the tutorials. The resulting web published tutorials\footnote{\url{https://docs.gammapy.org/0.8/tutorials.html}} provide links to versioned \textit{playground} spaces in myBinder \citep{project_jupyter-proc-scipy-2018}, where they may be executed on-line in virtual environments hosted in the myBinder infrastructure. These environments are built with the help of a \textit{Dockerfile} placed at the top level of the Github repository. 

The user may retrieve specific \textit{tutorial bundles} using the {\tt gammapy download} command. One \textit{tutorial bundle} is composed of a \textit{conda} file environment, Jupyter notebooks, and the datasets needed to reproduce them. For each of these bundles we specify, in centralized index lookup files, the required computing environment, which tutorials and datasets to provide, and where to fetch them from. Deterministic computing environments are defined in the form of \textit{conda} configuration files, with pinned version numbers for each dependency package. 

\section{Command line tools and functionalities}

We have developed a set of tools for users to download the \textit{tutorial bundles}, and for documentation maintainers to prepare and validate them. The {\tt gammapy download} command shown in Figure \ref{fig2}, provides users with the means to retrieve a \textit{tutorial bundle} or any tutorial-related asset (i.e. dataset or Jupyter notebook) for a specific version of the Gammapy package, so they can activate and use it at their will.

On the other side, the {\tt gammapy jupyter} command  provides documentation maintainers with a tool to seamlessly manage and integrate Jupyter notebooks into the documentation. This command provides functionalities for execution validation, code formatting, output cells stripping and straightforward execution of the notebooks. 

\articlefigure{P6-1_f1.eps}{fig1}{Technical set-up for building and shipping \textit{tutorial bundles}. The maintainers build the documentation from stripped Jupyter notebooks using Sphinx. The resulting published tutorials provide links to access myBinder \textit{playground} spaces, where the notebooks may be executed on-line. They may be also downloaded to the user desktop, together with the \textit{conda} virtual environment and the datasets needed.}

\articlefigure{P6-1_f2.eps}{fig2}{Users may retrieve versioned \textit{tutorial bundles}, or specific related digital assets like notebooks and datasets, using the {\tt gammapy download} command.}

\section{Conclussions}
We have presented a novel set-up to publish and distribute executable documentation version coupled with the code base of the Gammapy pakage. These \textit{tutorial bundles} are composed of a \textit{conda} virtual environment, Jupyter notebooks, and the datasets needed to reproduce them. The web published tutorials may be also executed on-line for learning purposes in myBinder \textit{playground} spaces. We find that this set-up greatly improves the user experience for a software in prototyping phase, where maintenance and delivery of the \textit{tutorial bundles} are done with on purpose command line tools. 

\acknowledgements The authors would like to thank the following projects and services that are used in the exposed work:  \textit{Git\footnote{\url{https://git-scm.com}}}, \textit{GitHub\footnote{\url{https://github.com}}}, \textit{Travis CI\footnote{\url{https://travis-ci.com}}}, \textit{Anaconda\footnote{\url{https://www.anaconda.com}}}, \textit{conda\footnote{\url{https://conda.io}}}, \textit{conda-forge\footnote{\url{https://conda-forge.org}}}, \textit{Sphinx\footnote{\url{https://www.sphinx-doc.org}}}, \textit{nbsphinx\footnote{\url{https://github.com/spatialaudio/nbsphinx}}}, and \textit{Black\footnote{\url{https://github.com/ambv/black}}}.


\bibliography{P6-1}  % For BibTex

\end{document}
